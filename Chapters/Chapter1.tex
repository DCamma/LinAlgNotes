\chapter{Vectors}
A real number can be represented by a point on a line, which is a 2-dimensional space, $\R$
\begin{center}
\begin{tikzpicture}
\draw[->] (0,0) -- (6,0);
\newcommand{\offset}{0.1}
\foreach \x in {0,...,5}
	\draw (\x,\offset) -- (\x,-\offset);

\foreach \x in {0,...,3}
		\draw (\x,-0.1) node[anchor=north] {\x};
\draw (4.5,-0.1) node[anchor=north] {4.5};	
\draw (4.5,0) node[] {x};	
\end{tikzpicture}
\end{center}

a pair of real numbers can be represented by a point on a plane, which is a 2-dimensional space, $\R^2$
\begin{center}
\begin{tikzpicture}[scale=0.5]
axis
	\draw[->] (-0.5,0) -- coordinate (x axis mid) (6,0) node[anchor=west]{$x_1$};
    	\draw[->] (0,-0.5) -- coordinate (y axis mid) (0,6) node[anchor=south]{$x_2$};;
    	%ticks
	\newcommand{\offset}{0.1}

    	\foreach \x in {0,...,5}
     		\draw (\x,\offset) -- (\x,-\offset);
    	
\foreach \y in {0,...,5}
     		\draw (\offset,\y) -- (-\offset,\y);
\draw[fill=black] (4,2) circle (0.05) node[anchor=west]{$\colvec{2}{4}{2}$};
\end{tikzpicture}
\end{center}


a triplet od real numbers can be represented by a point in 3D space, $\R^3$
\missingfigure{3D drawing}

\begin{definition}
	A vector is an ordered collection of $n$ numbers
\end{definition}

\begin{notation}
Usually vectors are given by letters, such as $u,v,w$. In textbooks vectors are written with bold font. In handwriting vectors are often written with a right arrow on top, such as $\overrightarrow{u}$. We will underline vectors, like so: $\underline{u}$.	
\end{notation}

\begin{definition}
Let us consider vector $\ul{u}\in\R^n$. The $i-$th component of vector \[\ul{u}=\colvec{3}{u_1}{\vdots}{u_n}\] is $u_i$
\end{definition}

\begin{example}
\[\ul{u}=\colvec{3}{3}{7}{11}\in\R^3\Rightarrow u_1=3, u_2=7, u_3=11 \]	
\end{example}

\begin{definition}
Let us consider vectors $\ul{u} \in\R^n$ and $\ul{v} \in\R^n$. Vector $\ul{w} \in\R^n$ is a sum of $\ul{u}$ and $\ul{v}$, $\ul{w}=\ul{u}+\ul{v}$, if $w_i=u_i+v_i$ for all $i=1,\dots,n$
\end{definition}

\begin{example}
\begin{enumerate}
	\item \[\ul{u}=\colvec{3}{3}{5}{1}, \ul{v}=\colvec{3}{-1}{0}{1}, \ul{w}=\ul{u}+\ul{v}=\colvec{3}{3+(-1)}{5+0}{1+1}=\colvec{3}{2}{5}{2}\]	
\item \[\ul{u}=\colvec{3}{3}{9}{-2}, \ul{v}=\colvec{4}{1}{2}{3}{0}\] $\uv+\vv$ is not defined! Both vectors should have the same number of components.
\end{enumerate}
\end{example}

\begin{definition}
\begin{enumerate}
	\item Vectors $\uv\in\R^n$ and $\vv\in\R^n$ are equal, if $u_i=v_i$ for all $i=1,\dots,n$
	\item A scalar is just another name for real number
	\item Let us consider a scalar $\alpha\in\R$ and vector $\uv \in\R^n$. A product of $\alpha$ and $\uv$ is defined as: \[\alpha\uv = \alpha\cdot\colvec{3}{u_1}{\vdots}{v_n} = \colvec{3}{\alpha\cdot u_1}{\vdots}{\alpha\cdot v_n}\]
\end{enumerate}
\end{definition}

\begin{example}
\[\alpha = 3, \uv = \colvec{4}{-1}{2}{5}{7}\Rightarrow \alpha\cdot\uv\colvec{4}{3\cdot -1}{3\cdot 2}{3\cdot 5}{3\cdot 7} = \colvec{4}{-3}{6}{15}{21} \]	
\end{example}

\begin{definition}
	Let us consider scalars $\alpha$ and $\beta$, and vectors $\uv\in\R^n$ and $\vv\in\R^n$. A sum of $\alpha\uv + \beta\cdot\vv$ is called a linear combination of vectors $\uv$ and $\vv$.
\end{definition}

\begin{example}
\begin{enumerate}
	\item \[2\cdot \colvec{3}{-1}{3}{5} + 3\cdot \colvec{3}{7}{2}{1} + 5\cdot \colvec{3}{1}{0}{-1} = \colvec{3}{24}{12}{8}\]	
	\item \[\uv - \vv = 1\cdot \uv +(-1)\cdot \vv = \colvec{3}{u_1-v_1}{\vdots}{u_i-v_i}\]
	\item \[\uv - \uv = \colvec{3}{u_1-u_1}{\vdots}{u_i-u_i}=\ul{0} \]
\end{enumerate}
\end{example}

\begin{definition}
	Vector $\uv\in\R^n$ is called a zero vector if all $u_i = 0$, $i=1,\dots,n$. The zero vector is often written as $\ul{0}\in\R^n$
\end{definition}

\section{Graphic representation of vectors and vector operations}
A vector can be represented in the following way: 
\begin{enumerate}
	\item An ordered collection of numbers, $\uv = \colvec{2}{3}{5}$
	\item As an arrow in space \begin{center}
\begin{tikzpicture}[scale=0.7]
	\draw[->] (-0.5,0) -- coordinate (x axis mid) (4,0) node[anchor=west]{$x_1$};
    	\draw[->] (0,-0.5) -- coordinate (y axis mid) (0,4) node[anchor=south]{$x_2$};;
    	%ticks
	\newcommand{\offset}{0.1}

    	\foreach \x in {0,...,3}
     		\draw (\x,\offset) -- (\x,-\offset);
    	
\foreach \y in {0,...,3}
     		\draw (\offset,\y) -- (-\offset,\y);

\draw[->] (0,0)--(3,1)node[anchor=west]{$\uv = \colvec{2}{3}{1}$};
\end{tikzpicture}

	\end{center}

	\item A vector is a point in space, the endpoint of a vector from the origin. 
\end{enumerate}
Let us consider vectors $\uv = \colvec{2}{3}{1}$, $\vv = \colvec{2}{-1}{2}$ and $\wv = \uv + \vv =\colvec{2}{2}{3}$

\begin{center}
\begin{tikzpicture}
	\draw[<->] (-2.5,0) -- coordinate (x axis mid) (4,0) node[anchor=west]{$x_1$};
    	\draw[->] (0,0) -- coordinate (y axis mid) (0,4) node[anchor=south]{$x_2$};;
    	%ticks
	\newcommand{\offset}{0.1}

    	\foreach \x in {-2,...,3}{
     		\draw (\x,\offset) -- (\x,-\offset);
		\draw (\x,-0.1) node[anchor=north] {\x};
		}
		
    	
\foreach \y in {0,...,3}{
     		\draw (\offset,\y) -- (-\offset,\y);
		\ifthenelse{ \equal{\y}{0} }{}{
		\draw (-0.1,\y) node[anchor=east] {\y};
		}
		}

\draw[dashed] (2,3)--(3,1);
\draw[dashed] (-1,2)--(2,3);
\draw[->] (0,0)--(3,1)node[anchor=west]{$\uv$};
\draw[->] (0,0)--(2,3)node[anchor=west]{$\wv$};
\draw[->] (0,0)--(-1,2)node[anchor=east]{$\vv$};
\end{tikzpicture}
\end{center}


Let us consider vector $\uv = \colvec{2}{3}{1}$. What is $2\cdot \uv$? We can calculate as follows:
\[
2\cdot \uv = 2\cdot\colvec{2}{3}{1} = \colvec{2}{6}{2}
\]
We stretch vector $\uv$ 2 times along the line defined by vector $\uv$. What is $-\uv$? Simply reverse the direction. What will be the representation of $\alpha\uv$ for all possible values of $\alpha$? An endless line

\begin{center}
\begin{tikzpicture}[scale=0.75]
	\draw[<->] (-2,0) -- coordinate (x axis mid) (8,0) node[anchor=west]{$x_1$};
    	\draw[->] (0,-0.5) -- coordinate (y axis mid) (0,4) node[anchor=south]{$x_2$};;
    	%ticks

\draw[->,thick] (0,0)--(3,1);
\draw (1.5,0.5) node[anchor=south] {$\uv$};
\draw (4.4,1.5) node[anchor=south] {$2\cdot\uv$};
\draw[->] (0,0)--(6,2);
\draw[] (-3,-1)--(9,3);
\end{tikzpicture}
\end{center}



Let us consider two vectors $\uv\in\R^2$ and $\vv\in\R^2$. What will be the representation of all linear combinations of $\uv$ and $\vv$, $\alpha\uv+\beta\vv$
\begin{enumerate}
	\item Plane: 
\begin{center}
\begin{tikzpicture}[scale=0.75]
	\draw[<->] (-2,0) -- coordinate (x axis mid) (8,0) node[anchor=west]{$x_1$};
    	\draw[->] (0,-0.5) -- coordinate (y axis mid) (0,4) node[anchor=south]{$x_2$};;
    	%ticks

\draw[->,thick] (0,0)--(3,1);
\draw (1.5,0.5) node[anchor=south] {$\uv$};
\draw[->,thick] (0,0)--(1,2);
\draw (0.5,1) node[anchor=west] {$\vv$};
\draw (4.4,1.5) node[anchor=south] {$2\cdot\uv$};
\draw[->] (0,0)--(6,2);
\draw[] (-3,-1)--(9,3);
\draw[] (-2,1)--(10,5);
\draw (4.2,3.2) node[anchor=east] {$\alpha\cdot\uv + 1\cdot\vv$};
\end{tikzpicture}

\end{center}

\item Line: $\uv$ and $\vv$ are on the same line.\\Note: Consider $\uv,\vv\in\R^n$. $\uv$ and $\vv$ are on the same line if there exists scalars $\alpha$ and $\beta$ such that $\alpha\uv + \beta\vv = \ul{0}$, when $\alpha$ and $\beta\not=0$ 
\item Point: if $\uv=\ul{0}$ and $\vv=\ul{0}\Rightarrow\alpha\uv+\beta\vv = \ul{0}$ 
\end{enumerate}
Consider $\vv,\uv$. They are on the same line if $\alpha\uv+\beta\vv = \ul{0}$ and $\alpha,\beta\not=0$
\section{Dot Product (Scalar product)}

\begin{definition}
Let us consider two vectors $\uv\in\R^n$ and $\vv\in\R^n$. The dot (or scalar) product of vectors $\uv$ and $\vv$ is defined as 
\[
\langle\uv,\vv\rangle = u_1v_1+u_2v_2+\dots+{\uv}_n{\vv}_n = \sum\limits^{n}_{i=1}u_iv_i
\]
\end{definition}

\begin{notation}
We will use $\langle\uv,\vv\rangle$ to denote the dot product, but sometimes $\uv\cdot\vv$ is used	
\end{notation}
\begin{example}
\begin{enumerate}
\item \[
\uv = \colvec{3}{1}{-1}{3}, \uv = \colvec{3}{0}{\frac{1}{2}}{-1}
\]
\[
\langle \uv,\vv\rangle = 1\cdot 0 + (-1)\cdot\frac{1}{2} + 3\cdot (-1) = -3.5
\]
\item \[
\uv = \colvec{2}{1}{0}, \uv = \colvec{3}{0}{1}, \langle\uv,\vv\rangle = 0
\]
\end{enumerate}	
\end{example}
Let us consider $\R^2$. What is the set of all possible endpoints of unit vectors in $\R^2$, originating from the origin?

\todo[inline]{Fix positioning problem}
\begin{figure}[ht]
{
\begin{minipage}[t]{0.45\linewidth}
\begin{center}
\begin{tikzpicture}[scale=0.5]
\draw[<->] (-4,0) -- coordinate (x axis mid) (4,0) node[anchor=west]{$x_1$};
\draw[<->] (0,-4) -- coordinate (y axis mid) (0,4) node[anchor=south]{$x_2$};
\draw[] (0,0) circle (3);
\draw[] (0,0)--(1.665,2.5);
\draw (0.5,0) arc (0:57:0.5);
\draw (0.5,0) node[anchor=south west]{$\theta$};
\draw[dashed](1.67,2.5) -- (1.67,0) node[anchor=north]{$u_1$};
\draw[dashed](1.67,2.5) -- (0,2.5)node[anchor=east]{$u_2$};
\draw (0.86,1.25) node[anchor=east]{$\uv$};
\end{tikzpicture}
\end{center}
\end{minipage}
}
\hspace{0.5cm}
{
\begin{minipage}[t]{0.45\linewidth}
\begin{align*}
\uv &= \colvec{2}{u_1}{u_2}\\
\cos(\theta) &= \frac{u_1}{\norm{\uv}} = u_1\\
\sin(\theta) &= \frac{u_2}{\norm{\uv}} = u_2\\
\uv &= \colvec{2}{\cos(\theta)}{\sin(\theta)}
\end{align*}
\end{minipage}
}
\end{figure}
Now let us consider two unit vectors
\begin{figure}[ht]
{
\begin{minipage}[t]{0.45\linewidth}
MISSING FIGURE PAGE 5, MIDDLE
\end{minipage}
}
\hspace{0.5cm}
{
\begin{minipage}[t]{0.45\linewidth}
\begin{align*}
\langle\uv,\vv\rangle &= \cos(\theta)\cos(\varphi) + \sin(\theta)\sin(\varphi)\\
&= \cos(\theta- \varphi) = \cos(\psi)\\
&= \cos(\angle(\uv,\vv))
\end{align*}
\end{minipage}
}
\end{figure}
If $\uv \not=\ul{0}$ or $\vv \not=\ul{0}$ are not unit vectors we can find the angle between them as follows:
\todo[inline]{Add unit vectors underbrace Page 5}
\begin{align*}
\langle\uv,\vv\rangle &= \left\langle \norm{\uv}\cdot\frac{1}{\norm{\uv}} \cdot \uv,\norm{\vv}\cdot\frac{1}{\norm{\vv}} \cdot \vv\right\rangle\\
&= \norm{\uv}\norm{\vv}\left\langle \frac{1}{\norm{\uv}} \cdot \uv,\frac{1}{\norm{\vv}} \cdot \vv\right\rangle\\ 
&= \norm{\uv}\norm{\vv}\cos\left( \angle(\uv,\vv)\right)
\end{align*}


\begin{lemma}
If $\uv\not=\ul{0},\vv\not=\ul{0}, \uv\in\R^n,\vv\in\R^n$, then 
\[\cos\left(\angle(\uv,\vv) \right) = \frac{\langle \uv,\vv\rangle}{\norm{\uv}\norm{\vv}}\]
\end{lemma}

\section{Properties of dot product}
\begin{enumerate}
	\item $\langle \alpha\cdot\uv,\vv\rangle = \alpha\cdot\langle\uv,\vv\rangle$ for any $\alpha\in\R,\uv\in\R^n,\vv\in\R^n$. Proof:
\begin{align*}
\langle \alpha\cdot\uv,\vv\rangle &= (\alpha u_1)\cdot v_1 + \dots + (\alpha u_n)\cdot v_n\\
&= \alpha\cdot(u_1\cdot v_1+\dots+u_n\cdot v_n)\\
&= \alpha\cdot\langle\uv,\vv\rangle
\end{align*}
\item $\langle\uv,\alpha\vv\rangle = \alpha\langle\uv,\vv\rangle$ for any $\alpha\in\R,\uv,\vv\in\R^n$
\item $\langle\alpha\uv+\beta\vv,\wv\rangle = \alpha\cdot\langle\uv,\wv\rangle + \beta\langle\vv,\wv\rangle, \forall\alpha\in\R,\forall\uv,\vv\wv\in\R^n$
\end{enumerate}
\begin{example}
\missingfigure{Page 6, middle}	
\end{example}
Let us consider $\uv = \colvec{2}{3}{4}. \langle \uv,\uv\rangle = 3\cdot 3+4\cdot 4 = 9+16 = 25 = 5^2$
\begin{definition}
The length of vector $\uv\in\R^n, \norm{\uv}$, is defined as $\norm{\uv} = \sqrt{\langle\uv,\uv\rangle}$. Sometimes it is also called the Euclidian norm of $\uv$.
\end{definition}
\begin{definition}
A vector with length equal to 1 is called a unit vector
\end{definition}
If we take vector $\uv\not=\ul{0}$, how to make it a unit vector? We should multiply vector $\uv$ by $\frac{1}{\norm{\uv}}$, we will get $\frac{\uv}{\norm{\uv}} = $ unit vector.\\

In our previous example: $\uv = \colvec{2}{3}{4}$. Unit vector is then 
\[\frac{\uv}{\norm{\uv}} = \frac{1}{5}\cdot \colvec{2}{3}{4} = \colvec{2}{\frac{3}{5}}{\frac{4}{5}} = \colvec{2}{0.6}{0.8}\]
We got $\langle\uv,\vv\rangle = \norm{\uv}\norm{\vv}\cdot\cos\left( \angle(\uv,\vv)\right)$
. Let us take the absolute value of this
\[\abs{\langle\uv,\vv\rangle} = \norm{\uv}\norm{\vv}\cdot\abs{\cos\left( \angle(\uv,\vv)\right)}\]
Notice that $\abs{\cos\left( \angle(\uv,\vv)\right)}\leq 1$
\begin{lemma}
Cauchy Schwartz Inequality: for any $\uv\in\R^n$ and $\vv\in\R^n$
\[\abs{\langle\uv,\vv\rangle}\leq\norm{\uv}\norm{\vv}\]
\end{lemma}
Remark: It is easy to see that Cauchy - Schwartz inequality is correct also for zero vectors.