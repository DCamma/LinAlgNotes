\chapter{Vectors}
A real number can be represented by a point on a line, which is a 2-dimensional space, $\R$
\missingfigure{Straight line with points}

a pair of real numbers can be represented by a point on a plane, which is a 2-dimensional space, $\R^2$
\missingfigure{Plane with vector on it}
a triplet od real numbers can be represented by a point in 3D space, $\R^3$
\missingfigure{3D drawing}

\begin{definition}
	A vector is an ordered collection of $n$ numbers
\end{definition}

\begin{notation}
Usually vectors are given by letters, such as $u,v,w$. In textbooks vectors are written with bold font. In handwriting vectors are often written with a right arrow on top, such as $\overrightarrow{u}$. We will underline vectors, like so: $\underline{u}$.	
\end{notation}

\begin{definition}
Let us consider vector $\ul{u}\in\R^n$. The $i-$th component of vector \[\ul{u}=\colvec{3}{u_1}{\vdots}{u_n}\] is $u_i$
\end{definition}

\begin{example}
\[\ul{u}=\colvec{3}{3}{7}{11}\in\R^3\Rightarrow u_1=3, u_2=7, u_3=11 \]	
\end{example}

\begin{definition}
Let us consider vectors $\ul{u} \in\R^n$ and $\ul{v} \in\R^n$. Vector $\ul{w} \in\R^n$ is a sum of $\ul{u}$ and $\ul{v}$, $\ul{w}=\ul{u}+\ul{v}$, if $w_i=u_i+v_i$ for all $i=1,\dots,n$
\end{definition}

\begin{example}
\begin{enumerate}
	\item \[\ul{u}=\colvec{3}{3}{5}{1}, \ul{v}=\colvec{3}{-1}{0}{1}, \ul{w}=\ul{u}+\ul{v}=\colvec{3}{3+(-1)}{5+0}{1+1}=\colvec{3}{2}{5}{2}\]	
\item \[\ul{u}=\colvec{3}{3}{9}{-2}, \ul{v}=\colvec{4}{1}{2}{3}{0}\] $\uv+\vv$ is not defined! Both vectors should have the same number of components.
\end{enumerate}
\end{example}

\begin{definition}
\begin{enumerate}
	\item Vectors $\uv\in\R^n$ and $\vv\in\R^n$ are equal, if $u_i=v_i$ for all $i=1,\dots,n$
	\item A scalar is just another name for real number
	\item Let us consider a scalar $\alpha\in\R$ and vector $\uv \in\R^n$. A product of $\alpha$ and $\uv$ is defined as: \[\alpha\uv = \alpha\cdot\colvec{3}{u_1}{\vdots}{v_n} = \colvec{3}{\alpha\cdot u_1}{\vdots}{\alpha\cdot v_n}\]
\end{enumerate}
\end{definition}

\begin{example}
\[\alpha = 3, \uv = \colvec{4}{-1}{2}{5}{7}\Rightarrow \alpha\cdot\uv\colvec{4}{3\cdot -1}{3\cdot 2}{3\cdot 5}{3\cdot 7} = \colvec{4}{-3}{6}{15}{21} \]	
\end{example}

\begin{definition}
	Let us consider scalars $\alpha$ and $\beta$, and vectors $\uv\in\R^n$ and $\vv\in\R^n$. A sum of $\alpha\uv + \beta\cdot\vv$ is called a linear combination of vectors $\uv$ and $\vv$.
\end{definition}

\begin{example}
\begin{enumerate}
	\item \[2\cdot \colvec{3}{-1}{3}{5} + 3\cdot \colvec{3}{7}{2}{1} + 5\cdot \colvec{3}{1}{0}{-1} = \colvec{3}{24}{12}{8}\]	
	\item \[\uv - \vv = 1\cdot \uv +(-1)\cdot \vv = \colvec{3}{u_1-v_1}{\vdots}{u_i-v_i}\]
	\item \[\uv - \uv = \colvec{3}{u_1-u_1}{\vdots}{u_i-u_i}=\ul{0} \]
\end{enumerate}
\end{example}

\begin{definition}
	Vector $\uv\in\R^n$ is called a zero vector if all $u_i = 0$, $i=1,\dots,n$. The zero vector is often written as $\ul{0}\in\R^n$
\end{definition}

\section{Graphic representation of vectors and vector operations}
A vector can be represented in the following way: 
\begin{enumerate}
	\item An ordered collection of numbers, $\uv = \colvec{2}{3}{5}$
	\item As an arrow in space \missingfigure{Page 3, middle}
	\item A vector is a point in space, the endpoint of a vector from the origin. 
\end{enumerate}
Let us consider vectors $\uv = \colvec{2}{3}{1}$, $\vv = \colvec{2}{-1}{2}$ and $\wv = \uv + \vv =\colvec{2}{2}{3}$
\missingfigure{Page 3, middle}
Let us consider vector $\uv = \colvec{2}{3}{1}$. What is $2\cdot \uv$? We can calculate as follows:
\[
2\cdot \uv = 2\cdot\colvec{2}{3}{1} = \colvec{2}{6}{2}
\]
We stretch vector $\uv$ 2 times along the line defined by vector $\uv$. What is $-\uv$? Simply reverse the direction. What will be the representation of $\alpha\uv$ for all possible values of $\alpha$? An endless line
\missingfigure{Page 3, bottom}
Let us consider two vectors $\uv\in\R^2$ and $\vv\in\R^2$. What will be the representation of all linear combinations of $\uv$ and $\vv$, $\alpha\uv+\beta\vv$
\begin{enumerate}
	\item Plane: \missingfigure{Page 4, top}
\item Line: $\uv$ and $\vv$ are on the same line.\\Note: Consider $\uv,\vv\in\R^n$. $\uv$ and $\vv$ are on the same line if there exists scalars $\alpha$ and $\beta$ such that $\alpha\uv + \beta\vv = \ul{0}$, when $\alpha$ and $\beta\not=0$ 
\item Point: if $\uv=\ul{0}$ and $\vv=\ul{0}\Rightarrow\alpha\uv+\beta\vv = \ul{0}$ 
\end{enumerate}
Consider $\\vv,uv$. They are on the same line if $\alpha\uv+\beta\vv = \ul{0}$ and $\alpha,\beta\not=0$
\section{Dot Product (Scalar product)}

\begin{definition}
Let us consider two vectors $\uv\in\R^n$ and $\vv\in\R^n$. The dot (or scalar) product of vectors $\uv$ and $\vv$ is defined as 
\[
\langle\uv,\vv\rangle = u_1v_1+u_2v_2+\dots+{\uv}_n{\vv}_n = \sum\limits^{n}_{i=1}u_iv_i
\]
\end{definition}

\begin{notation}
We will use $\langle\uv,\vv\rangle$ to denote the dot product, but sometimes $\uv\cdot\vv$ is used	
\end{notation}
\begin{example}
\begin{enumerate}
\item \[
\uv = \colvec{3}{1}{-1}{3}, \uv = \colvec{3}{0}{\frac{1}{2}}{-1}
\]
\[
\langle \uv,\vv\rangle = 1\cdot 0 + (-1)\cdot\frac{1}{2} + 3\cdot (-1) = -3.5
\]
\item \[
\uv = \colvec{2}{1}{0}, \uv = \colvec{3}{0}{1}, \langle\uv,\vv\rangle = 0
\]
\end{enumerate}	
\end{example}
Let us consider $\R^2$. What is the set of all possible endpoints of unit vectors in $\R^2$, originating from the origin?


\begin{figure}[ht]
{
\begin{minipage}[t]{0.45\linewidth}
MISSING FIGURE PAGE 5, TOP
\end{minipage}
}
\hspace{0.5cm}
{
\begin{minipage}[t]{0.45\linewidth}
\begin{align*}
\uv &= \colvec{2}{u_1}{u_2}\\
\cos(\theta) &= \frac{u_1}{\norm{\uv}} = u_1\\
\sin(\theta) &= \frac{u_2}{\norm{\uv}} = u_2\\
\uv &= \colvec{2}{\cos(\theta)}{\sin(\theta)}
\end{align*}
\end{minipage}
}
\end{figure}
Now let us consider two unit vectors
\begin{figure}[ht]
{
\begin{minipage}[t]{0.45\linewidth}
MISSING FIGURE PAGE 5, MIDDLE
\end{minipage}
}
\hspace{0.5cm}
{
\begin{minipage}[t]{0.45\linewidth}
\begin{align*}
\langle\uv,\vv\rangle &= \cos(\theta)\cos(\varphi) + \sin(\theta)\sin(\varphi)\\
&= \cos(\theta- \varphi) = \cos(\psi)\\
&= \cos(\angle(\uv,\vv))
\end{align*}
\end{minipage}
}
\end{figure}
If $\uv \not=\ul{0}$ or $\vv \not=\ul{0}$ are not unit vectors we can find the angle between them as follows:
\todo[inline]{Add unit vectors underbrace Page 5}
\begin{align*}
\langle\uv,\vv\rangle &= \left\langle \norm{\uv}\cdot\frac{1}{\norm{\uv}} \cdot \uv,\norm{\vv}\cdot\frac{1}{\norm{\vv}} \cdot \vv\right\rangle\\
&= \norm{\uv}\norm{\vv}\left\langle \frac{1}{\norm{\uv}} \cdot \uv,\frac{1}{\norm{\vv}} \cdot \vv\right\rangle\\ 
&= \norm{\uv}\norm{\vv}\cos\left( \angle(\uv,\vv)\right)
\end{align*}


\begin{lemma}
If $\uv\not=\ul{0},\vv\not=\ul{0}, \uv\in\R^n,\vv\in\R^n$, then 
\[\cos\left(\angle(\uv,\vv) \right) = \frac{\langle \uv,\vv\rangle}{\norm{\uv}\norm{\vv}}\]
\end{lemma}

%%%%%%%% Top of page 6 %%%%%%%%%
\section{Properties of dot product}
