\chapter{Gaussian Elimination}
\begin{example}
\[
A = \begin{pmatrix}
2 & 4 & -2 \\
4 & 9 & -3\\
-2 & -3 & 7
\end{pmatrix}. \ul{b}=\begin{pmatrix}
2\\
8\\
10
\end{pmatrix}, A\ul{x}= b
\]	
We can write this as a system of equations:
\[
\begin{rightalignedcases}
2x_1 + 4x_2-2x_3 = 2\\
4x_1 +9x_2 -3x_3 = 8\\
-2x_1 -3x_2 7x_3 = 10\\	
\end{rightalignedcases}
\]
We can multiply equation 1 by $-\frac{a_{21}}{a_{11}}= -\frac{4}{2} = -2$, and add to equation 2. This is equivalent to multiplying $A\ul{x} = \ul{b}$ by $E_{21}\left( -\frac{a_{21}}{a_{11}}\right)$ on the left. 

\begin{align*}
\begin{rightalignedcases}
2x_1 + 4x_2-2x_3 = 2\\
4x_1 +9x_2 -3x_3 = 8\\
-2x_1 -3x_2 7x_3 = 10\\	
\end{rightalignedcases}
\Leftrightarrow E_{21}\left( -\frac{a_{21}}{a_{11}}\right)\times A\ul{x} &= E_{21}\left( -\frac{a_{21}}{a_{11}}\right)\ul{b}\\
E_{21}\left( -\frac{a_{21}}{a_{11}}\right) &= \begin{pmatrix}
1 & 0 & 0\\
-2 & 1 & 0\\
0 & 0 & 1
\end{pmatrix}
\end{align*}
\[
\begin{rightalignedcases}
2x_1 + 4x_2-2x_3 = 2\\
x_2+x_3= 4\\
x_2+5x_3=12\\	
\end{rightalignedcases}
\]
\begin{align*}
\Leftrightarrow E_{31}\left( -\frac{a_{31}}{a_{11}}\right)E_{21}\left( -\frac{a_{21}}{a_{11}}\right)\times A\ul{x} &= E_{31}\left( -\frac{a_{31}}{a_{11}}\right)E_{21}\left( -\frac{a_{21}}{a_{11}}\right)\ul{b}\\
E_{31}\left( -\frac{a_{31}}{a_{11}}\right) &= \begin{pmatrix}
1 & 0 & 0\\
0 & 1 & 0\\
0 & 0 & 1
\end{pmatrix}
\end{align*}
We are done with the first column. Let us denote the resulting matrix by $A^{(1)}$
\[
\begin{rightalignedcases}
2x_1 + 4x_2-2x_3 = 2\\
x_2+x_3= 4\\
4x_3 = 8\\	
\end{rightalignedcases}
\]
\begin{align*}
\Leftrightarrow &E_{32}\left( -\frac{a_{32}^{(1)}}{a_{22}^{(1)}}\right)E_{31}\left( -\frac{a_{31}}{a_{11}}\right)E_{21}\left( -\frac{a_{21}}{a_{11}}\right)\times \underbrace{A\ul{x}}_{\ul{b}}\\
&E_{32}\left( -\frac{a_{32}^{(1)}}{a_{22}^{(1)}}\right) = \begin{pmatrix}
1 & 0 & 0\\
0 & 1 & 0\\
0 & -1 & 1
\end{pmatrix}
\end{align*}
We are done with the second column, so we can denote the resulting matrix by $A^{(2)}$.\\

In fact, we got an upper triangular matrix. We can solve it using backward compatibility. Let us denote 
\[
E_{32}\left( -\frac{a_{32}^{(1)}}{a_{22}^{(1)}}\right)E_{31}\left( -\frac{a_{31}}{a_{11}}\right)E_{21}\left( -\frac{a_{21}}{a_{11}}\right) = U
\]
where $U$ is the upper triangular matrix. Then the inverse of it is
\begin{align*}
&\left[ E_{32}\left( -\frac{a_{32}^{(1)}}{a_{22}^{(1)}}\right)E_{31}\left( -\frac{a_{31}}{a_{11}}\right)E_{21}\left( -\frac{a_{21}}{a_{11}}\right) \right]^{-1}\\
=& E_{21}\left( -\frac{a_{21}}{a_{11}}\right) E_{31}\left( -\frac{a_{31}}{a_{11}}\right)E_{32}\left( -\frac{a_{32}^{(1)}}{a_{22}^{(1)}}\right)\\
A=& \underbrace{E_{21}\left( -\frac{a_{21}}{a_{11}}\right) E_{31}\left( -\frac{a_{31}}{a_{11}}\right)E_{32}\left( -\frac{a_{32}^{(1)}}{a_{22}^{(1)}}\right)}_{L}\cdot U
\end{align*}
All matrices $E_{xx}(x)$ are lower triangular $\to$ the product is also lower triangular ($A = l\cdot U$). So using Gaussian elimination, we represented $A$ as a product of lower and upper triangular matrices
\[
A\ul{x} = \ul{b} \Rightarrow LU\ul{x} = \ul{b}
\]
\end{example}

Let us denote $U\ul{x}$ by $\ul{y}$, then we get
\[
\begin{rightalignedcases}
l\ul{y} = \ul{b} & \text{Solve by forward substitution, find $\ul{y}$}\\
U\ul{x} = \ul{y} & \text{Solve by backward substitution}\\
\end{rightalignedcases}
\]
\begin{remark}
Gaussian elimination works if all elements $a_{11},a_{22}^{(1)},a_{33}^{(2)},\dots, ,a_{ii}^{(i-1)}$ are non-zero! These elements are called \underline{PIVOT} elements.
\end{remark}
\begin{example}
\begin{align*}
&\begin{rightalignedcases}
2x_1 + 4x_2 - 2x_3 = 2\\
4x_1 + 8x_2 - 3x_3 = 6\\
-2x_1 -3x_2 + 7x_3 = 10\\
\end{rightalignedcases} \Leftrightarrow A\ul{x} = \ul{b}\\
&\begin{rightalignedcases}
2x_1 + 4x_2 - 2x_3 = 2\\
x_3 = 2\\
-2x_1 -3x_2 +7x_3 = 10\\
\end{rightalignedcases}\Leftrightarrow E_{21}\left( -\frac{a_{21}}{a_{11}}\right) A\ul{x} = E_{21}\left( -\frac{a_{21}}{a_{11}}\right)\\
&\begin{rightalignedcases}
2x_1 + 4x_2 - 2x_3 = 2\\
x_3 = 2\\
x_2 + 5x_3 = 12\\
\end{rightalignedcases}\Leftrightarrow E_{31}\left( -\frac{a_{31}}{a_{11}}\right)  A\ul{x} = E_{31}\left( -\frac{a_{31}}{a_{11}}\right) E_{21}\left( -\frac{a_{21}}{a_{11}}\right) \ul{b}\\
\end{align*}
We denote the resulting matrix by $A^{(1)}$. In order to proceed we need $a_{22}^{(1)} \not=0$. Let us consider matrix $P_{pq}$-matrix, which you get from identity matrix by exchanging rows $p$ and $q$. It is easy to show that $P_{pq}\cdot A$ is equal to matrix $A$ with rows $p$ and $q$ exchanged. 
\begin{definition}
Permutation matrix $P$ is an identity matrix with rows in any order.
\end{definition}
\begin{remark}
$P^{-1} = P$. The product of permutation on matrices is a permutation matrix. 	
\end{remark}\\

We want to exchange rows 2 and 3. We need to multiply by the permutation matrix $P_{23}$
\[
\begin{rightalignedcases}
2x_1 + 4x_2 -2x_3 = 2\\
x_2 + 5x_3 = 12\\
x_3 = 2
\end{rightalignedcases}
\]
\begin{align*}
\Leftrightarrow &P_{23}\cdot E_{31}\left( -\frac{a_{31}}{a_{11}}\right)E_{21}\left( -\frac{a_{21}}{a_{11}}\right)A\ul{x} \\
= & P_{23}\cdot E_{31}\left( -\frac{a_{31}}{a_{11}}\right)E_{21}\left( -\frac{a_{21}}{a_{11}}\right)\ul{b}
\end{align*} 
In general, the Gaussian elimination proceeds like this:
\[
E_{xx}\dots E_{xx}P_{xx}E_{xx} \dots  E_{xx}A\ul{x} = E_{xx}\dots E_{xx}P_{xx}E_{xx}\dots E_{xx}\ul{b}
\]
Turns out, that we can exchange the rows, or in other words multiply $A$ by $(P_{xx}\dots P_{xx}$ before doing the Gaussian elimination
\begin{align*}
\underbrace{\underbrace{(E_{xx}\dots E_{xx})}_{E}\underbrace{(P_{xx}\dots P_{xx})}_{P}A}_{U}\ul{x} &= (E_{xx}\dots E_{xx})(P_{xx}\dots P_{xx})\ul{b} \\
EPA &= U\\
PA &= E^{-1} U = LU \leftarrow \text{Lower triangular}
\end{align*}
\end{example}
\begin{theorem}
There exists permutation matrix $P$, such that $PA = LU$. The only necessary condition for that is that $A^{-1}$ exists.
\end{theorem}
\section{Matrix Transposition}
\begin{definition}
Let us consider matrix $A\in\R^{m,n}$. Matrix $B\in\R^{n,m}$ is called the transpose of $A$ if $(B)_{ij} = (A)_{ji}, i=1\dots n, j=1\dots n$
\end{definition}
\begin{notation}
Usually the transpose of $A$ is written as $A^T$	
\end{notation}
\begin{example}
\[
A = \begin{pmatrix}
2 & 3\\
4 & 5\\
6 & 7\\
9 & 10
\end{pmatrix} \in\R^{4,2} \Rightarrow A = \begin{pmatrix}
2 & 4 & 6 & 9\\
3 & 5 & 7 & 10
\end{pmatrix} \in\R^{2,4}
\]	
\end{example}
\begin{properties}
\begin{enumerate}
\item $\left( A^T\right)^T = A$
\item $\left( A+B\right)^T = A^T + B^T$
\item $\left( AB\right)^T = B^T\cdot A^T$
\item $\left( A^T\right)^{-1} = \left( A^{-1}\right)^{T}$
\end{enumerate}
\end{properties}
\begin{proof}
\begin{enumerate}
\item[3.]  
\begin{align*}
A\in\R^{m,n} &= \begin{pmatrix}
-\text{row 1} \to \\
\vdots \\
-\text{row $n$} \to \\
\end{pmatrix},B\in\R^{n,l} = \begin{pmatrix}
\text{\rotatebox{270}{-col 1$\to$}} & \dots & \text{\rotatebox{270}{-col n$\to$}} 
\end{pmatrix}\\
(AB)_{ij} &= \langle \text{row $i$ of $A$, column $j$ of $B$}\rangle \\
\left((AB)^{T}\right)_{pq} &= (AB)_{qp} =  \langle \text{row $q$ of $A$, column $p$ of $B$}\rangle \\
B^T &= \begin{pmatrix}
-\text{col 1} \to \\
\vdots \\
-\text{col $n$} \to \\
\end{pmatrix},A^T = \begin{pmatrix}
\text{\rotatebox{270}{-col 1$\to$}} & \dots & \text{\rotatebox{270}{-col n$\to$}} 
\end{pmatrix} \\
\left( B^T A^T\right)_{pq} &= \langle \text{column $p$ of $B$, row $q$ of $A$}\rangle\\
\Rightarrow \left( (AB)^T\right)_{pq} &= \left( B^T A^T\right)_{pq}; p=1,\dots, l; q=1,\dots,m.\\
\Rightarrow (AB)^T &= B^T A^T
\end{align*} 
\item[4.] Assume that $A\in \R^{n,n},\exists A^{-1}$
\begin{align*}
AA^{-1} &= I \to (AA^{-1})^T = \left(A^{-1}\right)^T \cdot A^T = I^T = I\\
A^{-1}A &= I \to (A^{-1}A)^T = A^T \cdot \left(A^{-1}\right)^T  = I^T = I\\
\left( A^T\right)^{-1} &= \left( A^{-1}\right)^{T}
\end{align*}
\end{enumerate}	
\end{proof}
Let us consider vector $\uv = \colvec{3}{u_1}{\vdots }{u_n}\in \R^{n,1}$ - column vector. Then $\uv^T\in \R^{1,n} = (u_1\dots u_n)$ - row vector. Let us also consider $\vv = \colvec{3}{v_1}{\vdots }{v_n}\in \R^{n,1}$. Then
\begin{align*}
\uv^T\cdot \vv = (u_1\dots u_n)\cdot \colvec{3}{v_1}{\vdots }{v_n} = u_1v_1 + u_2v_2 + \dots + u_nv_n = \langle \uv,\vv\rangle\\
\vv \cdot \uv^T = n\times n \text{ matrix}
\end{align*}

\begin{definition}
Matrix $A$ is called symmetric if $A^t = A$. Matrix $A$ should be a square matrix, $A\in\R^{n,n}$
\begin{align*}
\text{e.g. }& A= \begin{pmatrix}
0 & 3 \\
3 & 4
\end{pmatrix} \to A^T = \begin{pmatrix}
0 & 3 \\
3 & 4
\end{pmatrix} \Rightarrow A^T = A\\
\text{e.g. }& A = I\in \R^{n,n} \to I^T = I
\end{align*}
\end{definition}







