\documentclass[a4paper]{article}

\input{Packages.tex}
\newcommand{\R}{\mathbb{R}} % set R with \R
\newcommand{\N}{\mathbb{N}} % set N with \N
\newcommand{\C}{\mathbb{C}} % set C with \C
\newcommand{\uv}{\underline{u}}
\newcommand{\vv}{\underline{v}}
\newcommand{\wv}{\underline{w}}


\newcommand{\ul}[1]{\underline{#1}} 

\newenvironment{definition}{\begin{framed}\centerline{\textbf{Definition}}\noindent\hspace{-1.1mm}}{\end{framed}}
\newenvironment{notation}{\subsubsection*{Notation}}{\begin{flushright}$\square$\end{flushright}}
\newenvironment{example}{\subsubsection*{Example}}{}%\begin{flushright}$\square$\end{flushright}}
\newenvironment{lemma}{\subsubsection*{Lemma}}{}

\DeclareMathOperator{\arccosh}{arccosh}%Defined here since it doesn't exist
\DeclareMathOperator{\arcsinh}{arcsinh}%Defined here since it doesn't exist
\DeclareMathOperator{\arctanh}{arctanh}%Defined here since it doesn't exist
\newcommand{\verteq}[0]{\begin{turn}{90} $=$\end{turn}}%added vertical equal sign
\DeclareMathOperator{\Hess}{Hess}%Defined here since it doesn't exist
\DeclareMathOperator{\rot}{rot}%Defined here since it doesn't exist
\DeclareMathOperator{\Bild}{Bild}%Defined here since it doesn't exist
\DeclareMathOperator{\grad}{grad}%Defined here since it doesn't exist

\newcommand{\defeq}{\vcentcolon=}% nicer :=
\newcommand{\eqdef}{=\vcentcolon}% nicer =:

\DeclarePairedDelimiter\abs{\lvert}{\rvert} % nice |x|
\DeclarePairedDelimiter\norm{\lVert}{\rVert} % nice ||x|| \norm{x} => ||x||
% Swap the definition of \abs* and \norm*, so that \abs
% and \norm resizes the size of the brackets, and the 
% starred version does not.
\makeatletter
\let\oldabs\abs
\def\abs{\@ifstar{\oldabs}{\oldabs*}}
\let\oldnorm\norm
\def\norm{\@ifstar{\oldnorm}{\oldnorm*}}
\makeatother


% Source: http://tex.stackexchange.com/questions/2705/typesetting-column-vector. use with \colvec{5}{a}{b}{c}{d}{e}
\newcount\colveccount
\newcommand*\colvec[1]{
        \global\colveccount#1
        \begin{pmatrix}
        \colvecnext
}
\def\colvecnext#1{
        #1
        \global\advance\colveccount-1
        \ifnum\colveccount>0
                \\
                \expandafter\colvecnext
        \else
                \end{pmatrix}
        \fi
}
\usepackage{fancyvrb}
\begin{document}
This document aims to be a small example of how to transcribe the Linear Algebra notes, so to keep a consistency among sections written by different people.

\section{File structure}
The project is organized in the following way:
\begin{Verbatim}[obeytabs,tabsize=4]
Chapters
	Chapter1.tex
	Chapter2.tex
	. . .
Other
	CustomEnvironments.tex
	Sample.pdf
	Sample.tex
	packages.tex
README.md
RawMaterial
	Core.pdf
	Lecture_17_03_2015.pdf
	. . .
main.pdf
main.tex
\end{Verbatim}

If you need to add files (for example chapter files) please keep the same folder structure. 
\section{Available commands}
There are a few environments available for you to use, these allow you to apply custom styles without too much effort. These include:
\begin{itemize}
	\item definition
\item notation
\item example
\item lemma
\item properties
\end{itemize}
In order to see exactly what they do, and to have a more complete overview of the helper methods, take a look at \verb+Other/CustomEnvironments.tex+\\

These are quite easy to use, for example if you were to write a definition you can simply write the following:
\begin{verbatim}
\begin{definition}
Some kind of definition. This definition also includes some math:
\[
\sum\limits_{i=0}^{n}i^2
\]	
\end{definition}
\end{verbatim}
This code will produce the following output:
\begin{definition}
Some kind of definition. This definition also includes some math:
\[
\sum\limits_{i=0}^{n}i^2
\]	
\end{definition}
\section{Packages}
Packages are all contained in the file \verb+Other/packages.tex+. If you need to include a package, add it there. Please keep packages to a minimum, and include if and only if there is no other way to do a certain thing.
\section{Styling}
So as to keep the style consistent, please use the following pointers when typing:
\begin{itemize}
	\item Math mode and inline mode have to be used in a logical way. It is up to you when to use what, but look at the other chapters and keep it consistent.
\item Mathmode has to be activated using \verb+\[...\]+, not \verb+$$...$$+. This is due to some internal issues with how \LaTeX~handles math mode when used with the dollar signs.
\item 
\end{itemize}


\end{document}
