\newcommand{\R}{\mathbb{R}}
\newcommand{\N}{\mathbb{N}}
\newcommand{\C}{\mathbb{C}}
\newcommand{\Lm}{\mathcal{L}}
\newcommand{\uv}{\underline{u}}
\newcommand{\vv}{\underline{v}}
\newcommand{\wv}{\underline{w}}
\newcommand{\ul}[1]{\underline{#1}} 

% ================ CUSTOM ENVIRONMENTS ================
\newenvironment{definition}{\begin{mdframed}\centerline{\textbf{Definition}}\vspace{2mm}\noindent\hspace{-1.1mm}}{\end{mdframed}}
\newenvironment{notation}{\subsubsection*{Notation}}{\begin{flushright}$\square$\end{flushright}}
\newenvironment{example}{\subsubsection*{Example}}{}%\begin{flushright}$\square$\end{flushright}}
\newenvironment{lemma}{\subsubsection*{Lemma}}{}
\newenvironment{properties}{\subsubsection*{Properties}}{}
\newenvironment{remark}{\subsubsection*{\underline{Remark:}}}{}
\newenvironment{theorem}{\subsubsection*{Theorem}}{}
\newenvironment{recap}{\subsubsection*{Recap}}{}
\newenvironment{note}{\subsubsection*{Note}}{}
\newenvironment{proof}{\subsubsection*{Proof}}{\begin{flushright}$\square$\end{flushright}}

% ================ MATH OPERATORS ================
\DeclareMathOperator{\arccosh}{arccosh}
\DeclareMathOperator{\rank}{rank}
\DeclareMathOperator{\arcsinh}{arcsinh}
\DeclareMathOperator{\arctanh}{arctanh}

% Abs and Norm
\DeclarePairedDelimiter\abs{\lvert}{\rvert} % nice |x|
\DeclarePairedDelimiter\norm{\lVert}{\rVert} % nice ||x|| \norm{x} => ||x||
% Swap the definition of \abs* and \norm*, so that \abs
% and \norm resizes the size of the brackets, and the 
% starred version does not.
\makeatletter
\let\oldabs\abs
\def\abs{\@ifstar{\oldabs}{\oldabs*}}
\let\oldnorm\norm
\def\norm{\@ifstar{\oldnorm}{\oldnorm*}}
\makeatother

% Vector method
% Source: http://tex.stackexchange.com/questions/2705/typesetting-column-vector. use with \colvec{5}{a}{b}{c}{d}{e}
\newcount\colveccount
\newcommand*\colvec[1]{
        \global\colveccount#1
        \begin{pmatrix}
        \colvecnext
}
\def\colvecnext#1{
        #1
        \global\advance\colveccount-1
        \ifnum\colveccount>0
                \\
                \expandafter\colvecnext
        \else
                \end{pmatrix}
        \fi
}



% http://tex.stackexchange.com/questions/102460/underbraces-in-matrix-divided-in-blocks
\newcommand\undermat[2]{%
  \makebox[0pt][l]{$\smash{\underbrace{\phantom{%
    \begin{matrix}#2\end{matrix}}}_{\text{$#1$}}}$}#2}
